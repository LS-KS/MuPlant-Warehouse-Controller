%% Generated by Sphinx.
\def\sphinxdocclass{report}
\documentclass[letterpaper,10pt,english]{sphinxmanual}
\ifdefined\pdfpxdimen
   \let\sphinxpxdimen\pdfpxdimen\else\newdimen\sphinxpxdimen
\fi \sphinxpxdimen=.75bp\relax
\ifdefined\pdfimageresolution
    \pdfimageresolution= \numexpr \dimexpr1in\relax/\sphinxpxdimen\relax
\fi
%% let collapsible pdf bookmarks panel have high depth per default
\PassOptionsToPackage{bookmarksdepth=5}{hyperref}

\PassOptionsToPackage{booktabs}{sphinx}
\PassOptionsToPackage{colorrows}{sphinx}

\PassOptionsToPackage{warn}{textcomp}
\usepackage[utf8]{inputenc}
\ifdefined\DeclareUnicodeCharacter
% support both utf8 and utf8x syntaxes
  \ifdefined\DeclareUnicodeCharacterAsOptional
    \def\sphinxDUC#1{\DeclareUnicodeCharacter{"#1}}
  \else
    \let\sphinxDUC\DeclareUnicodeCharacter
  \fi
  \sphinxDUC{00A0}{\nobreakspace}
  \sphinxDUC{2500}{\sphinxunichar{2500}}
  \sphinxDUC{2502}{\sphinxunichar{2502}}
  \sphinxDUC{2514}{\sphinxunichar{2514}}
  \sphinxDUC{251C}{\sphinxunichar{251C}}
  \sphinxDUC{2572}{\textbackslash}
\fi
\usepackage{cmap}
\usepackage[T1]{fontenc}
\usepackage{amsmath,amssymb,amstext}
\usepackage{babel}



\usepackage{tgtermes}
\usepackage{tgheros}
\renewcommand{\ttdefault}{txtt}



\usepackage[Bjarne]{fncychap}
\usepackage{sphinx}

\fvset{fontsize=auto}
\usepackage{geometry}


% Include hyperref last.
\usepackage{hyperref}
% Fix anchor placement for figures with captions.
\usepackage{hypcap}% it must be loaded after hyperref.
% Set up styles of URL: it should be placed after hyperref.
\urlstyle{same}

\addto\captionsenglish{\renewcommand{\contentsname}{Table of Contents}}

\usepackage{sphinxmessages}
\setcounter{tocdepth}{1}



\title{muPlantPython}
\date{May 25, 2023}
\release{0.1}
\author{Lennart Schink}
\newcommand{\sphinxlogo}{\vbox{}}
\renewcommand{\releasename}{Release}
\makeindex
\begin{document}

\ifdefined\shorthandoff
  \ifnum\catcode`\=\string=\active\shorthandoff{=}\fi
  \ifnum\catcode`\"=\active\shorthandoff{"}\fi
\fi

\pagestyle{empty}
\sphinxmaketitle
\pagestyle{plain}
\sphinxtableofcontents
\pagestyle{normal}
\phantomsection\label{\detokenize{index::doc}}


\sphinxAtStartPar
This Software is part of muPlant Project of University of Kassel.
It implements basic functions for WareHouse Management.


\chapter{Communication Standards}
\label{\detokenize{index:communication-standards}}\begin{itemize}
\item {} 
\sphinxAtStartPar
TCP/IP for communication with ABB Robot

\item {} 
\sphinxAtStartPar
OPC UA with other muPlant stations

\item {} 
\sphinxAtStartPar
RFID to communicate with turtle bots

\item {} 
\sphinxAtStartPar
uEye Camera with openCV and arUco markers for automated storage detection

\end{itemize}


\chapter{Learnings from earlier studies}
\label{\detokenize{index:learnings-from-earlier-studies}}
\sphinxAtStartPar
Sebastian Hübler has published hie practical studies in 2019. He evaluated methods to detect cups
in muPlant storage by using two different cameras.
Regarding the detection with arUco markers he made some helpful analysis:
\begin{itemize}
\item {} 
\sphinxAtStartPar
low resolution leads to better detection results

\item {} 
\sphinxAtStartPar
minimal/ maximal distance uEye tp marker: 17mm/ 745mm

\item {} 
\sphinxAtStartPar
ambient light has significant influence

\item {} 
\sphinxAtStartPar
auto focus maybe a big issue

\item {} 
\sphinxAtStartPar
detection best works while robotic arm is without movement

\end{itemize}


\section{Further thoughts:}
\label{\detokenize{index:further-thoughts}}\begin{itemize}
\item {} 
\sphinxAtStartPar
if ambient light and other surcumstances are not good:

\item {} 
\sphinxAtStartPar
reduce image size to criticl area only

\item {} 
\sphinxAtStartPar
calibrate location of camera to maximize image size reduction

\item {} 
\sphinxAtStartPar
better detection with custom filter which smoothes and increases contrast?

\item {} 
\sphinxAtStartPar
other camera which has no autofocus

\end{itemize}


\chapter{Indices and tables}
\label{\detokenize{index:indices-and-tables}}

\section{Getting started}
\label{\detokenize{index:getting-started}}\begin{itemize}
\item {} 
\sphinxAtStartPar
install Python 3.11

\item {} 
\sphinxAtStartPar
install numpy

\item {} 
\sphinxAtStartPar
install asynchua

\item {} 
\sphinxAtStartPar
install pyside6

\item {} 
\sphinxAtStartPar
install websocket

\item {} 
\sphinxAtStartPar
install opencv

\end{itemize}


\section{Modules}
\label{\detokenize{index:modules}}
\sphinxstepscope


\subsection{Modules and Scripts}
\label{\detokenize{modules:modules-and-scripts}}\label{\detokenize{modules::doc}}
\sphinxAtStartPar
This list contains all created Modules and scripts created for this software.

\phantomsection\label{\detokenize{modules:module-main}}\index{module@\spxentry{module}!main@\spxentry{main}}\index{main@\spxentry{main}!module@\spxentry{module}}
\sphinxAtStartPar
This is the entrancee file of Python\sphinxhyphen{}Implementation of muPlant Warehouse Manager.
Author: L.Schink
Date: 11.05.2023

\phantomsection\label{\detokenize{modules:module-src.cameraApplication.cameraProcessing}}\index{module@\spxentry{module}!src.cameraApplication.cameraProcessing@\spxentry{src.cameraApplication.cameraProcessing}}\index{src.cameraApplication.cameraProcessing@\spxentry{src.cameraApplication.cameraProcessing}!module@\spxentry{module}}
\sphinxAtStartPar
This Python File implements the logic to recognize arUco markers.
class VideoThread inherits from QThread class. So image capture and image processing code is in seperated thread.
Processed images are provided to qml by using class videoPlayer which inherits from QQuickImageProvider.
\index{VideoPlayer (class in src.cameraApplication.cameraProcessing)@\spxentry{VideoPlayer}\spxextra{class in src.cameraApplication.cameraProcessing}}

\begin{fulllineitems}
\phantomsection\label{\detokenize{modules:src.cameraApplication.cameraProcessing.VideoPlayer}}
\pysigstartsignatures
\pysigline{\sphinxbfcode{\sphinxupquote{class\DUrole{w,w}{  }}}\sphinxcode{\sphinxupquote{src.cameraApplication.cameraProcessing.}}\sphinxbfcode{\sphinxupquote{VideoPlayer}}}
\pysigstopsignatures\index{requestImage() (src.cameraApplication.cameraProcessing.VideoPlayer method)@\spxentry{requestImage()}\spxextra{src.cameraApplication.cameraProcessing.VideoPlayer method}}

\begin{fulllineitems}
\phantomsection\label{\detokenize{modules:src.cameraApplication.cameraProcessing.VideoPlayer.requestImage}}
\pysigstartsignatures
\pysiglinewithargsret{\sphinxbfcode{\sphinxupquote{requestImage}}}{\sphinxparam{\DUrole{n,n}{id}}, \sphinxparam{\DUrole{n,n}{size}}, \sphinxparam{\DUrole{n,n}{requestedSize}}}{}
\pysigstopsignatures
\sphinxAtStartPar
This function overrides requestImage from inherited class.
:param id: necessary identifier to switch between images. Can be any value. Implemented as boolean value which is toggled
everytime when imageChanged is emitted form a JavaScript \sphinxhyphen{} function in CameraApplicationMain.qml
:param size:
:param requestedSize:
:return: returns QImage object in RGBA color format

\end{fulllineitems}

\index{start() (src.cameraApplication.cameraProcessing.VideoPlayer method)@\spxentry{start()}\spxextra{src.cameraApplication.cameraProcessing.VideoPlayer method}}

\begin{fulllineitems}
\phantomsection\label{\detokenize{modules:src.cameraApplication.cameraProcessing.VideoPlayer.start}}
\pysigstartsignatures
\pysiglinewithargsret{\sphinxbfcode{\sphinxupquote{start}}}{}{}
\pysigstopsignatures
\sphinxAtStartPar
Overrides start method of inherited class QQuickImageProvider. It is a Slot and called from QML Button of CameraAppMain.qml
:return: this method returns nothing.

\end{fulllineitems}

\index{stop() (src.cameraApplication.cameraProcessing.VideoPlayer method)@\spxentry{stop()}\spxextra{src.cameraApplication.cameraProcessing.VideoPlayer method}}

\begin{fulllineitems}
\phantomsection\label{\detokenize{modules:src.cameraApplication.cameraProcessing.VideoPlayer.stop}}
\pysigstartsignatures
\pysiglinewithargsret{\sphinxbfcode{\sphinxupquote{stop}}}{}{}
\pysigstopsignatures
\sphinxAtStartPar
Overrides stop method of inherited class QQuickImageProvider. It is a Slot and called from QML Button of CameraAppMain.qml
:return: this method returns nothing

\end{fulllineitems}

\index{toggleDetection() (src.cameraApplication.cameraProcessing.VideoPlayer method)@\spxentry{toggleDetection()}\spxextra{src.cameraApplication.cameraProcessing.VideoPlayer method}}

\begin{fulllineitems}
\phantomsection\label{\detokenize{modules:src.cameraApplication.cameraProcessing.VideoPlayer.toggleDetection}}
\pysigstartsignatures
\pysiglinewithargsret{\sphinxbfcode{\sphinxupquote{toggleDetection}}}{}{}
\pysigstopsignatures
\sphinxAtStartPar
Toggles detection field of VideoThread object. Enables / disables feature detection in VideoThread’s run method.
It is a Slot and called from QML Button of CameraAppMain.qml
:return: This method returns nothing

\end{fulllineitems}

\index{updateImage() (src.cameraApplication.cameraProcessing.VideoPlayer method)@\spxentry{updateImage()}\spxextra{src.cameraApplication.cameraProcessing.VideoPlayer method}}

\begin{fulllineitems}
\phantomsection\label{\detokenize{modules:src.cameraApplication.cameraProcessing.VideoPlayer.updateImage}}
\pysigstartsignatures
\pysiglinewithargsret{\sphinxbfcode{\sphinxupquote{updateImage}}}{\sphinxparam{\DUrole{n,n}{frame}}}{}
\pysigstopsignatures
\sphinxAtStartPar
Implements connection between VideoThread and VideoPlayer. If VideoThread emits a new image this Slot is called.
stores emitted image in self.image and emits image to QQmlEngine
:param frame: QImage which is emitted from run\sphinxhyphen{}method in VideoThread object.
:return: this method returns nothing but emits signal to QQmlEngine

\end{fulllineitems}


\end{fulllineitems}

\index{VideoThread (class in src.cameraApplication.cameraProcessing)@\spxentry{VideoThread}\spxextra{class in src.cameraApplication.cameraProcessing}}

\begin{fulllineitems}
\phantomsection\label{\detokenize{modules:src.cameraApplication.cameraProcessing.VideoThread}}
\pysigstartsignatures
\pysiglinewithargsret{\sphinxbfcode{\sphinxupquote{class\DUrole{w,w}{  }}}\sphinxcode{\sphinxupquote{src.cameraApplication.cameraProcessing.}}\sphinxbfcode{\sphinxupquote{VideoThread}}}{\sphinxparam{\DUrole{n,n}{parent}\DUrole{o,o}{=}\DUrole{default_value}{None}}}{}
\pysigstopsignatures\index{capture (src.cameraApplication.cameraProcessing.VideoThread attribute)@\spxentry{capture}\spxextra{src.cameraApplication.cameraProcessing.VideoThread attribute}}

\begin{fulllineitems}
\phantomsection\label{\detokenize{modules:src.cameraApplication.cameraProcessing.VideoThread.capture}}
\pysigstartsignatures
\pysigline{\sphinxbfcode{\sphinxupquote{capture}}}
\pysigstopsignatures
\sphinxAtStartPar
initializes the first camera device.

\end{fulllineitems}

\index{detect() (src.cameraApplication.cameraProcessing.VideoThread method)@\spxentry{detect()}\spxextra{src.cameraApplication.cameraProcessing.VideoThread method}}

\begin{fulllineitems}
\phantomsection\label{\detokenize{modules:src.cameraApplication.cameraProcessing.VideoThread.detect}}
\pysigstartsignatures
\pysiglinewithargsret{\sphinxbfcode{\sphinxupquote{detect}}}{}{}
\pysigstopsignatures
\sphinxAtStartPar
enables/disables detection in run\sphinxhyphen{}method

\end{fulllineitems}

\index{detecting (src.cameraApplication.cameraProcessing.VideoThread attribute)@\spxentry{detecting}\spxextra{src.cameraApplication.cameraProcessing.VideoThread attribute}}

\begin{fulllineitems}
\phantomsection\label{\detokenize{modules:src.cameraApplication.cameraProcessing.VideoThread.detecting}}
\pysigstartsignatures
\pysigline{\sphinxbfcode{\sphinxupquote{detecting}}}
\pysigstopsignatures
\sphinxAtStartPar
enables/disables feature detection

\end{fulllineitems}

\index{faceCascade (src.cameraApplication.cameraProcessing.VideoThread attribute)@\spxentry{faceCascade}\spxextra{src.cameraApplication.cameraProcessing.VideoThread attribute}}

\begin{fulllineitems}
\phantomsection\label{\detokenize{modules:src.cameraApplication.cameraProcessing.VideoThread.faceCascade}}
\pysigstartsignatures
\pysigline{\sphinxbfcode{\sphinxupquote{faceCascade}}}
\pysigstopsignatures
\sphinxAtStartPar
initialize haar cascade face detection.. just that there is some image processing

\end{fulllineitems}

\index{frameChanged (src.cameraApplication.cameraProcessing.VideoThread attribute)@\spxentry{frameChanged}\spxextra{src.cameraApplication.cameraProcessing.VideoThread attribute}}

\begin{fulllineitems}
\phantomsection\label{\detokenize{modules:src.cameraApplication.cameraProcessing.VideoThread.frameChanged}}
\pysigstartsignatures
\pysigline{\sphinxbfcode{\sphinxupquote{frameChanged}}}
\pysigstopsignatures
\sphinxAtStartPar
Signal which is emitted when a new image is ready for QQuickImageProvider

\end{fulllineitems}

\index{quit() (src.cameraApplication.cameraProcessing.VideoThread method)@\spxentry{quit()}\spxextra{src.cameraApplication.cameraProcessing.VideoThread method}}

\begin{fulllineitems}
\phantomsection\label{\detokenize{modules:src.cameraApplication.cameraProcessing.VideoThread.quit}}
\pysigstartsignatures
\pysiglinewithargsret{\sphinxbfcode{\sphinxupquote{quit}}}{}{}
\pysigstopsignatures
\sphinxAtStartPar
Necessary Implementation of inherited class to quit existing thread.

\end{fulllineitems}

\index{run() (src.cameraApplication.cameraProcessing.VideoThread method)@\spxentry{run()}\spxextra{src.cameraApplication.cameraProcessing.VideoThread method}}

\begin{fulllineitems}
\phantomsection\label{\detokenize{modules:src.cameraApplication.cameraProcessing.VideoThread.run}}
\pysigstartsignatures
\pysiglinewithargsret{\sphinxbfcode{\sphinxupquote{run}}}{}{}
\pysigstopsignatures
\sphinxAtStartPar
This Method reads the camera sensor and performs necessary image processing.
Converts processed image to Qt’s QImage class and emits Signal with QImage

\end{fulllineitems}

\index{running (src.cameraApplication.cameraProcessing.VideoThread attribute)@\spxentry{running}\spxextra{src.cameraApplication.cameraProcessing.VideoThread attribute}}

\begin{fulllineitems}
\phantomsection\label{\detokenize{modules:src.cameraApplication.cameraProcessing.VideoThread.running}}
\pysigstartsignatures
\pysigline{\sphinxbfcode{\sphinxupquote{running}}}
\pysigstopsignatures
\sphinxAtStartPar
run variable for while loop in run() function

\end{fulllineitems}

\index{start() (src.cameraApplication.cameraProcessing.VideoThread method)@\spxentry{start()}\spxextra{src.cameraApplication.cameraProcessing.VideoThread method}}

\begin{fulllineitems}
\phantomsection\label{\detokenize{modules:src.cameraApplication.cameraProcessing.VideoThread.start}}
\pysigstartsignatures
\pysiglinewithargsret{\sphinxbfcode{\sphinxupquote{start}}}{}{}
\pysigstopsignatures
\sphinxAtStartPar
Necessary Implementation of inherited class to quit existing thread.

\end{fulllineitems}


\end{fulllineitems}

\phantomsection\label{\detokenize{modules:module-src.controller.EventlogController}}\index{module@\spxentry{module}!src.controller.EventlogController@\spxentry{src.controller.EventlogController}}\index{src.controller.EventlogController@\spxentry{src.controller.EventlogController}!module@\spxentry{module}}\index{EventlogController (class in src.controller.EventlogController)@\spxentry{EventlogController}\spxextra{class in src.controller.EventlogController}}

\begin{fulllineitems}
\phantomsection\label{\detokenize{modules:src.controller.EventlogController.EventlogController}}
\pysigstartsignatures
\pysigline{\sphinxbfcode{\sphinxupquote{class\DUrole{w,w}{  }}}\sphinxcode{\sphinxupquote{src.controller.EventlogController.}}\sphinxbfcode{\sphinxupquote{EventlogController}}}
\pysigstopsignatures
\end{fulllineitems}

\phantomsection\label{\detokenize{modules:module-src.controller.InventoryController}}\index{module@\spxentry{module}!src.controller.InventoryController@\spxentry{src.controller.InventoryController}}\index{src.controller.InventoryController@\spxentry{src.controller.InventoryController}!module@\spxentry{module}}\index{InventoryController (class in src.controller.InventoryController)@\spxentry{InventoryController}\spxextra{class in src.controller.InventoryController}}

\begin{fulllineitems}
\phantomsection\label{\detokenize{modules:src.controller.InventoryController.InventoryController}}
\pysigstartsignatures
\pysiglinewithargsret{\sphinxbfcode{\sphinxupquote{class\DUrole{w,w}{  }}}\sphinxcode{\sphinxupquote{src.controller.InventoryController.}}\sphinxbfcode{\sphinxupquote{InventoryController}}}{\sphinxparam{\DUrole{n,n}{model}\DUrole{p,p}{:}\DUrole{w,w}{  }\DUrole{n,n}{{\hyperref[\detokenize{modules:src.model.InventoryModel.InventoryModel}]{\sphinxcrossref{InventoryModel}}}}\DUrole{w,w}{  }\DUrole{o,o}{=}\DUrole{w,w}{  }\DUrole{default_value}{None}}, \sphinxparam{\DUrole{n,n}{parent}\DUrole{o,o}{=}\DUrole{default_value}{None}}, \sphinxparam{\DUrole{n,n}{eventcontroller}\DUrole{p,p}{:}\DUrole{w,w}{  }\DUrole{n,n}{{\hyperref[\detokenize{modules:src.controller.EventlogController.EventlogController}]{\sphinxcrossref{EventlogController}}}}\DUrole{w,w}{  }\DUrole{o,o}{=}\DUrole{w,w}{  }\DUrole{default_value}{None}}, \sphinxparam{\DUrole{n,n}{productlist}\DUrole{p,p}{:}\DUrole{w,w}{  }\DUrole{n,n}{{\hyperref[\detokenize{modules:src.model.ProductListModel.ProductListModel}]{\sphinxcrossref{ProductListModel}}}}\DUrole{w,w}{  }\DUrole{o,o}{=}\DUrole{w,w}{  }\DUrole{default_value}{None}}}{}
\pysigstopsignatures\index{changeStorage() (src.controller.InventoryController.InventoryController method)@\spxentry{changeStorage()}\spxextra{src.controller.InventoryController.InventoryController method}}

\begin{fulllineitems}
\phantomsection\label{\detokenize{modules:src.controller.InventoryController.InventoryController.changeStorage}}
\pysigstartsignatures
\pysiglinewithargsret{\sphinxbfcode{\sphinxupquote{changeStorage}}}{\sphinxparam{\DUrole{n,n}{storage}}, \sphinxparam{\DUrole{n,n}{slot}}, \sphinxparam{\DUrole{n,n}{cupID}}, \sphinxparam{\DUrole{n,n}{productID}}}{}
\pysigstopsignatures
\sphinxAtStartPar
Takes Data from Override Storage Dialog from Storage.qml
Decodes Storage ID ‘L1’ to L’18’ in row / col and checks for ValueErrors.
changes InventoryModel Data depending on entries.

\end{fulllineitems}

\index{loadStorage() (src.controller.InventoryController.InventoryController method)@\spxentry{loadStorage()}\spxextra{src.controller.InventoryController.InventoryController method}}

\begin{fulllineitems}
\phantomsection\label{\detokenize{modules:src.controller.InventoryController.InventoryController.loadStorage}}
\pysigstartsignatures
\pysiglinewithargsret{\sphinxbfcode{\sphinxupquote{loadStorage}}}{\sphinxparam{\DUrole{n,n}{storage}\DUrole{p,p}{:}\DUrole{w,w}{  }\DUrole{n,n}{str}}, \sphinxparam{\DUrole{n,n}{slot}\DUrole{p,p}{:}\DUrole{w,w}{  }\DUrole{n,n}{str}}}{}
\pysigstopsignatures
\sphinxAtStartPar
Takes Data from Override Storage Dialog from Storage.qml
Decodes Storage ID ‘L1’ to L’18’ in row / col and checks for ValueErrors.
returns productslot, cup ID and productListindex.

\end{fulllineitems}


\end{fulllineitems}

\phantomsection\label{\detokenize{modules:module-src.controller.websocketController}}\index{module@\spxentry{module}!src.controller.websocketController@\spxentry{src.controller.websocketController}}\index{src.controller.websocketController@\spxentry{src.controller.websocketController}!module@\spxentry{module}}\index{WebsocketController (class in src.controller.websocketController)@\spxentry{WebsocketController}\spxextra{class in src.controller.websocketController}}

\begin{fulllineitems}
\phantomsection\label{\detokenize{modules:src.controller.websocketController.WebsocketController}}
\pysigstartsignatures
\pysiglinewithargsret{\sphinxbfcode{\sphinxupquote{class\DUrole{w,w}{  }}}\sphinxcode{\sphinxupquote{src.controller.websocketController.}}\sphinxbfcode{\sphinxupquote{WebsocketController}}}{\sphinxparam{\DUrole{n,n}{controller}\DUrole{p,p}{:}\DUrole{w,w}{  }\DUrole{n,n}{{\hyperref[\detokenize{modules:src.controller.EventlogController.EventlogController}]{\sphinxcrossref{EventlogController}}}}}, \sphinxparam{\DUrole{n,n}{parent}\DUrole{o,o}{=}\DUrole{default_value}{None}}}{}
\pysigstopsignatures
\end{fulllineitems}

\phantomsection\label{\detokenize{modules:module-src.model.InventoryModel}}\index{module@\spxentry{module}!src.model.InventoryModel@\spxentry{src.model.InventoryModel}}\index{src.model.InventoryModel@\spxentry{src.model.InventoryModel}!module@\spxentry{module}}\index{InventoryModel (class in src.model.InventoryModel)@\spxentry{InventoryModel}\spxextra{class in src.model.InventoryModel}}

\begin{fulllineitems}
\phantomsection\label{\detokenize{modules:src.model.InventoryModel.InventoryModel}}
\pysigstartsignatures
\pysiglinewithargsret{\sphinxbfcode{\sphinxupquote{class\DUrole{w,w}{  }}}\sphinxcode{\sphinxupquote{src.model.InventoryModel.}}\sphinxbfcode{\sphinxupquote{InventoryModel}}}{\sphinxparam{\DUrole{n,n}{storageData}}, \sphinxparam{\DUrole{n,n}{parent}\DUrole{o,o}{=}\DUrole{default_value}{None}}}{}
\pysigstopsignatures\index{columnCount() (src.model.InventoryModel.InventoryModel method)@\spxentry{columnCount()}\spxextra{src.model.InventoryModel.InventoryModel method}}

\begin{fulllineitems}
\phantomsection\label{\detokenize{modules:src.model.InventoryModel.InventoryModel.columnCount}}
\pysigstartsignatures
\pysiglinewithargsret{\sphinxbfcode{\sphinxupquote{columnCount}}}{\sphinxparam{\DUrole{n,n}{self}}, \sphinxparam{\DUrole{n,n}{parent}\DUrole{p,p}{:}\DUrole{w,w}{  }\DUrole{n,n}{PySide6.QtCore.QModelIndex\DUrole{w,w}{  }\DUrole{p,p}{|}\DUrole{w,w}{  }PySide6.QtCore.QPersistentModelIndex}\DUrole{w,w}{  }\DUrole{o,o}{=}\DUrole{w,w}{  }\DUrole{default_value}{Invalid(PySide6.QtCore.QModelIndex)}}}{{ $\rightarrow$ int}}
\pysigstopsignatures
\end{fulllineitems}

\index{data() (src.model.InventoryModel.InventoryModel method)@\spxentry{data()}\spxextra{src.model.InventoryModel.InventoryModel method}}

\begin{fulllineitems}
\phantomsection\label{\detokenize{modules:src.model.InventoryModel.InventoryModel.data}}
\pysigstartsignatures
\pysiglinewithargsret{\sphinxbfcode{\sphinxupquote{data}}}{\sphinxparam{\DUrole{n,n}{self}}, \sphinxparam{\DUrole{n,n}{index}\DUrole{p,p}{:}\DUrole{w,w}{  }\DUrole{n,n}{PySide6.QtCore.QModelIndex\DUrole{w,w}{  }\DUrole{p,p}{|}\DUrole{w,w}{  }PySide6.QtCore.QPersistentModelIndex}}, \sphinxparam{\DUrole{n,n}{role}\DUrole{p,p}{:}\DUrole{w,w}{  }\DUrole{n,n}{int}\DUrole{w,w}{  }\DUrole{o,o}{=}\DUrole{w,w}{  }\DUrole{default_value}{Instance(Qt.DisplayRole)}}}{{ $\rightarrow$ Any}}
\pysigstopsignatures
\end{fulllineitems}

\index{roleNames() (src.model.InventoryModel.InventoryModel method)@\spxentry{roleNames()}\spxextra{src.model.InventoryModel.InventoryModel method}}

\begin{fulllineitems}
\phantomsection\label{\detokenize{modules:src.model.InventoryModel.InventoryModel.roleNames}}
\pysigstartsignatures
\pysiglinewithargsret{\sphinxbfcode{\sphinxupquote{roleNames}}}{\sphinxparam{\DUrole{n,n}{self}}}{{ $\rightarrow$ Dict\DUrole{p,p}{{[}}int\DUrole{p,p}{,}\DUrole{w,w}{  }PySide6.QtCore.QByteArray\DUrole{p,p}{{]}}}}
\pysigstopsignatures
\end{fulllineitems}

\index{rowCount() (src.model.InventoryModel.InventoryModel method)@\spxentry{rowCount()}\spxextra{src.model.InventoryModel.InventoryModel method}}

\begin{fulllineitems}
\phantomsection\label{\detokenize{modules:src.model.InventoryModel.InventoryModel.rowCount}}
\pysigstartsignatures
\pysiglinewithargsret{\sphinxbfcode{\sphinxupquote{rowCount}}}{\sphinxparam{\DUrole{n,n}{self}}, \sphinxparam{\DUrole{n,n}{parent}\DUrole{p,p}{:}\DUrole{w,w}{  }\DUrole{n,n}{PySide6.QtCore.QModelIndex\DUrole{w,w}{  }\DUrole{p,p}{|}\DUrole{w,w}{  }PySide6.QtCore.QPersistentModelIndex}\DUrole{w,w}{  }\DUrole{o,o}{=}\DUrole{w,w}{  }\DUrole{default_value}{Invalid(PySide6.QtCore.QModelIndex)}}}{{ $\rightarrow$ int}}
\pysigstopsignatures
\end{fulllineitems}

\index{setData() (src.model.InventoryModel.InventoryModel method)@\spxentry{setData()}\spxextra{src.model.InventoryModel.InventoryModel method}}

\begin{fulllineitems}
\phantomsection\label{\detokenize{modules:src.model.InventoryModel.InventoryModel.setData}}
\pysigstartsignatures
\pysiglinewithargsret{\sphinxbfcode{\sphinxupquote{setData}}}{\sphinxparam{\DUrole{n,n}{self}}, \sphinxparam{\DUrole{n,n}{index}\DUrole{p,p}{:}\DUrole{w,w}{  }\DUrole{n,n}{PySide6.QtCore.QModelIndex\DUrole{w,w}{  }\DUrole{p,p}{|}\DUrole{w,w}{  }PySide6.QtCore.QPersistentModelIndex}}, \sphinxparam{\DUrole{n,n}{value}\DUrole{p,p}{:}\DUrole{w,w}{  }\DUrole{n,n}{Any}}, \sphinxparam{\DUrole{n,n}{role}\DUrole{p,p}{:}\DUrole{w,w}{  }\DUrole{n,n}{int}\DUrole{w,w}{  }\DUrole{o,o}{=}\DUrole{w,w}{  }\DUrole{default_value}{Instance(Qt.EditRole)}}}{{ $\rightarrow$ bool}}
\pysigstopsignatures
\end{fulllineitems}


\end{fulllineitems}

\phantomsection\label{\detokenize{modules:module-src.model.ProductListModel}}\index{module@\spxentry{module}!src.model.ProductListModel@\spxentry{src.model.ProductListModel}}\index{src.model.ProductListModel@\spxentry{src.model.ProductListModel}!module@\spxentry{module}}\index{ProductListModel (class in src.model.ProductListModel)@\spxentry{ProductListModel}\spxextra{class in src.model.ProductListModel}}

\begin{fulllineitems}
\phantomsection\label{\detokenize{modules:src.model.ProductListModel.ProductListModel}}
\pysigstartsignatures
\pysiglinewithargsret{\sphinxbfcode{\sphinxupquote{class\DUrole{w,w}{  }}}\sphinxcode{\sphinxupquote{src.model.ProductListModel.}}\sphinxbfcode{\sphinxupquote{ProductListModel}}}{\sphinxparam{\DUrole{n,n}{products}}, \sphinxparam{\DUrole{n,n}{parent}\DUrole{o,o}{=}\DUrole{default_value}{None}}}{}
\pysigstopsignatures\index{data() (src.model.ProductListModel.ProductListModel method)@\spxentry{data()}\spxextra{src.model.ProductListModel.ProductListModel method}}

\begin{fulllineitems}
\phantomsection\label{\detokenize{modules:src.model.ProductListModel.ProductListModel.data}}
\pysigstartsignatures
\pysiglinewithargsret{\sphinxbfcode{\sphinxupquote{data}}}{\sphinxparam{\DUrole{n,n}{index}}, \sphinxparam{\DUrole{n,n}{role}}}{}
\pysigstopsignatures
\sphinxAtStartPar
Returns an appropriate value for the requested data.
If the view requests an invalid index, an invalid variant is returned.
Any valid index that corresponds to a string in the list causes that
string to be returned
:param index:
:param role:
:return:

\end{fulllineitems}

\index{headerData() (src.model.ProductListModel.ProductListModel method)@\spxentry{headerData()}\spxextra{src.model.ProductListModel.ProductListModel method}}

\begin{fulllineitems}
\phantomsection\label{\detokenize{modules:src.model.ProductListModel.ProductListModel.headerData}}
\pysigstartsignatures
\pysiglinewithargsret{\sphinxbfcode{\sphinxupquote{headerData}}}{\sphinxparam{\DUrole{n,n}{section}}, \sphinxparam{\DUrole{n,n}{orientation}}, \sphinxparam{\DUrole{n,n}{role}\DUrole{o,o}{=}\DUrole{default_value}{ItemDataRole.DisplayRole}}}{}
\pysigstopsignatures
\sphinxAtStartPar
Returns the appropriate header string depending on the orientation of
the header and the section. If anything other than the display role is
requested, we return an invalid variant.

\end{fulllineitems}

\index{roleNames() (src.model.ProductListModel.ProductListModel method)@\spxentry{roleNames()}\spxextra{src.model.ProductListModel.ProductListModel method}}

\begin{fulllineitems}
\phantomsection\label{\detokenize{modules:src.model.ProductListModel.ProductListModel.roleNames}}
\pysigstartsignatures
\pysiglinewithargsret{\sphinxbfcode{\sphinxupquote{roleNames}}}{\sphinxparam{\DUrole{n,n}{self}}}{{ $\rightarrow$ Dict\DUrole{p,p}{{[}}int\DUrole{p,p}{,}\DUrole{w,w}{  }PySide6.QtCore.QByteArray\DUrole{p,p}{{]}}}}
\pysigstopsignatures
\end{fulllineitems}

\index{rowCount() (src.model.ProductListModel.ProductListModel method)@\spxentry{rowCount()}\spxextra{src.model.ProductListModel.ProductListModel method}}

\begin{fulllineitems}
\phantomsection\label{\detokenize{modules:src.model.ProductListModel.ProductListModel.rowCount}}
\pysigstartsignatures
\pysiglinewithargsret{\sphinxbfcode{\sphinxupquote{rowCount}}}{\sphinxparam{\DUrole{n,n}{self}}, \sphinxparam{\DUrole{n,n}{parent}\DUrole{p,p}{:}\DUrole{w,w}{  }\DUrole{n,n}{PySide6.QtCore.QModelIndex\DUrole{w,w}{  }\DUrole{p,p}{|}\DUrole{w,w}{  }PySide6.QtCore.QPersistentModelIndex}\DUrole{w,w}{  }\DUrole{o,o}{=}\DUrole{w,w}{  }\DUrole{default_value}{Invalid(PySide6.QtCore.QModelIndex)}}}{{ $\rightarrow$ int}}
\pysigstopsignatures
\end{fulllineitems}


\end{fulllineitems}

\phantomsection\label{\detokenize{modules:module-src.model.ProductSummaryListModel}}\index{module@\spxentry{module}!src.model.ProductSummaryListModel@\spxentry{src.model.ProductSummaryListModel}}\index{src.model.ProductSummaryListModel@\spxentry{src.model.ProductSummaryListModel}!module@\spxentry{module}}\index{InventoryFilterProxyModel (class in src.model.ProductSummaryListModel)@\spxentry{InventoryFilterProxyModel}\spxextra{class in src.model.ProductSummaryListModel}}

\begin{fulllineitems}
\phantomsection\label{\detokenize{modules:src.model.ProductSummaryListModel.InventoryFilterProxyModel}}
\pysigstartsignatures
\pysiglinewithargsret{\sphinxbfcode{\sphinxupquote{class\DUrole{w,w}{  }}}\sphinxcode{\sphinxupquote{src.model.ProductSummaryListModel.}}\sphinxbfcode{\sphinxupquote{InventoryFilterProxyModel}}}{\sphinxparam{\DUrole{n,n}{model}}, \sphinxparam{\DUrole{n,n}{parent}\DUrole{o,o}{=}\DUrole{default_value}{None}}}{}
\pysigstopsignatures\index{filterAcceptsRow() (src.model.ProductSummaryListModel.InventoryFilterProxyModel method)@\spxentry{filterAcceptsRow()}\spxextra{src.model.ProductSummaryListModel.InventoryFilterProxyModel method}}

\begin{fulllineitems}
\phantomsection\label{\detokenize{modules:src.model.ProductSummaryListModel.InventoryFilterProxyModel.filterAcceptsRow}}
\pysigstartsignatures
\pysiglinewithargsret{\sphinxbfcode{\sphinxupquote{filterAcceptsRow}}}{\sphinxparam{\DUrole{n,n}{self}}, \sphinxparam{\DUrole{n,n}{source\_row}\DUrole{p,p}{:}\DUrole{w,w}{  }\DUrole{n,n}{int}}, \sphinxparam{\DUrole{n,n}{source\_parent}\DUrole{p,p}{:}\DUrole{w,w}{  }\DUrole{n,n}{PySide6.QtCore.QModelIndex\DUrole{w,w}{  }\DUrole{p,p}{|}\DUrole{w,w}{  }PySide6.QtCore.QPersistentModelIndex}}}{{ $\rightarrow$ bool}}
\pysigstopsignatures
\end{fulllineitems}


\end{fulllineitems}

\index{ProductSummaryListModel (class in src.model.ProductSummaryListModel)@\spxentry{ProductSummaryListModel}\spxextra{class in src.model.ProductSummaryListModel}}

\begin{fulllineitems}
\phantomsection\label{\detokenize{modules:src.model.ProductSummaryListModel.ProductSummaryListModel}}
\pysigstartsignatures
\pysiglinewithargsret{\sphinxbfcode{\sphinxupquote{class\DUrole{w,w}{  }}}\sphinxcode{\sphinxupquote{src.model.ProductSummaryListModel.}}\sphinxbfcode{\sphinxupquote{ProductSummaryListModel}}}{\sphinxparam{\DUrole{n,n}{products}}, \sphinxparam{\DUrole{n,n}{parent}\DUrole{o,o}{=}\DUrole{default_value}{None}}}{}
\pysigstopsignatures\index{data() (src.model.ProductSummaryListModel.ProductSummaryListModel method)@\spxentry{data()}\spxextra{src.model.ProductSummaryListModel.ProductSummaryListModel method}}

\begin{fulllineitems}
\phantomsection\label{\detokenize{modules:src.model.ProductSummaryListModel.ProductSummaryListModel.data}}
\pysigstartsignatures
\pysiglinewithargsret{\sphinxbfcode{\sphinxupquote{data}}}{\sphinxparam{\DUrole{n,n}{index}}, \sphinxparam{\DUrole{n,n}{role}}}{}
\pysigstopsignatures
\sphinxAtStartPar
Returns an appropriate value for the requested data.
If the view requests an invalid index, an invalid variant is returned.
Any valid index that corresponds to a string in the list causes that
string to be returned.
:param index:
:param role:
:return:

\end{fulllineitems}

\index{headerData() (src.model.ProductSummaryListModel.ProductSummaryListModel method)@\spxentry{headerData()}\spxextra{src.model.ProductSummaryListModel.ProductSummaryListModel method}}

\begin{fulllineitems}
\phantomsection\label{\detokenize{modules:src.model.ProductSummaryListModel.ProductSummaryListModel.headerData}}
\pysigstartsignatures
\pysiglinewithargsret{\sphinxbfcode{\sphinxupquote{headerData}}}{\sphinxparam{\DUrole{n,n}{section}}, \sphinxparam{\DUrole{n,n}{orientation}}, \sphinxparam{\DUrole{n,n}{role}\DUrole{o,o}{=}\DUrole{default_value}{ItemDataRole.DisplayRole}}}{}
\pysigstopsignatures
\sphinxAtStartPar
Returns the appropriate header string depending on the orientation of
the header and the section. If anything other than the display role is
requested, we return an invalid variant
:param section:
:param orientation:
:param role:
:return:

\end{fulllineitems}

\index{roleNames() (src.model.ProductSummaryListModel.ProductSummaryListModel method)@\spxentry{roleNames()}\spxextra{src.model.ProductSummaryListModel.ProductSummaryListModel method}}

\begin{fulllineitems}
\phantomsection\label{\detokenize{modules:src.model.ProductSummaryListModel.ProductSummaryListModel.roleNames}}
\pysigstartsignatures
\pysiglinewithargsret{\sphinxbfcode{\sphinxupquote{roleNames}}}{\sphinxparam{\DUrole{n,n}{self}}}{{ $\rightarrow$ Dict\DUrole{p,p}{{[}}int\DUrole{p,p}{,}\DUrole{w,w}{  }PySide6.QtCore.QByteArray\DUrole{p,p}{{]}}}}
\pysigstopsignatures
\end{fulllineitems}

\index{rowCount() (src.model.ProductSummaryListModel.ProductSummaryListModel method)@\spxentry{rowCount()}\spxextra{src.model.ProductSummaryListModel.ProductSummaryListModel method}}

\begin{fulllineitems}
\phantomsection\label{\detokenize{modules:src.model.ProductSummaryListModel.ProductSummaryListModel.rowCount}}
\pysigstartsignatures
\pysiglinewithargsret{\sphinxbfcode{\sphinxupquote{rowCount}}}{\sphinxparam{\DUrole{n,n}{self}}, \sphinxparam{\DUrole{n,n}{parent}\DUrole{p,p}{:}\DUrole{w,w}{  }\DUrole{n,n}{PySide6.QtCore.QModelIndex\DUrole{w,w}{  }\DUrole{p,p}{|}\DUrole{w,w}{  }PySide6.QtCore.QPersistentModelIndex}\DUrole{w,w}{  }\DUrole{o,o}{=}\DUrole{w,w}{  }\DUrole{default_value}{Invalid(PySide6.QtCore.QModelIndex)}}}{{ $\rightarrow$ int}}
\pysigstopsignatures
\end{fulllineitems}


\end{fulllineitems}

\phantomsection\label{\detokenize{modules:module-src.opcua.opcuaClient}}\index{module@\spxentry{module}!src.opcua.opcuaClient@\spxentry{src.opcua.opcuaClient}}\index{src.opcua.opcuaClient@\spxentry{src.opcua.opcuaClient}!module@\spxentry{module}}\phantomsection\label{\detokenize{modules:module-src.websocket.websocketClient}}\index{module@\spxentry{module}!src.websocket.websocketClient@\spxentry{src.websocket.websocketClient}}\index{src.websocket.websocketClient@\spxentry{src.websocket.websocketClient}!module@\spxentry{module}}

\section{Contributions}
\label{\detokenize{index:contributions}}\begin{itemize}
\item {} 
\sphinxAtStartPar
Qt Project

\end{itemize}


\renewcommand{\indexname}{Python Module Index}
\begin{sphinxtheindex}
\let\bigletter\sphinxstyleindexlettergroup
\bigletter{m}
\item\relax\sphinxstyleindexentry{main}\sphinxstyleindexpageref{modules:\detokenize{module-main}}
\indexspace
\bigletter{s}
\item\relax\sphinxstyleindexentry{src.cameraApplication.cameraProcessing}\sphinxstyleindexpageref{modules:\detokenize{module-src.cameraApplication.cameraProcessing}}
\item\relax\sphinxstyleindexentry{src.controller.EventlogController}\sphinxstyleindexpageref{modules:\detokenize{module-src.controller.EventlogController}}
\item\relax\sphinxstyleindexentry{src.controller.InventoryController}\sphinxstyleindexpageref{modules:\detokenize{module-src.controller.InventoryController}}
\item\relax\sphinxstyleindexentry{src.controller.websocketController}\sphinxstyleindexpageref{modules:\detokenize{module-src.controller.websocketController}}
\item\relax\sphinxstyleindexentry{src.model.InventoryModel}\sphinxstyleindexpageref{modules:\detokenize{module-src.model.InventoryModel}}
\item\relax\sphinxstyleindexentry{src.model.ProductListModel}\sphinxstyleindexpageref{modules:\detokenize{module-src.model.ProductListModel}}
\item\relax\sphinxstyleindexentry{src.model.ProductSummaryListModel}\sphinxstyleindexpageref{modules:\detokenize{module-src.model.ProductSummaryListModel}}
\item\relax\sphinxstyleindexentry{src.opcua.opcuaClient}\sphinxstyleindexpageref{modules:\detokenize{module-src.opcua.opcuaClient}}
\item\relax\sphinxstyleindexentry{src.websocket.websocketClient}\sphinxstyleindexpageref{modules:\detokenize{module-src.websocket.websocketClient}}
\end{sphinxtheindex}

\renewcommand{\indexname}{Index}
\printindex
\end{document}
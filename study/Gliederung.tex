%! Author = lennartschink
%! Date = 29.05.23

% Preamble
\documentclass[11pt]{article}
\title {Analyse und Bewertung einer C\#-basierten Lagerverwaltungssoftware als Vorbereitung für die Migration in eine umfassende Python-Anwendung mit Fokus auf kameragestützte Validierungsprozesse}
\author{Lennart Schink}
\date{\today}
% Packages
\usepackage{amsmath}

% Document
\begin{document}

    \maketitle
    \tableofcontents
    \newpage

    \section{Einleitung}
    Die Seminiararbeit soll vor allem die Planung der Bachelorarbeit behandeln

    \newpage
    \section {Softwarearchitektur der bestehenden Software}

    \subsection {Lagerverwaltung 3.0}

    \subsection {RFID Server}

    \subsection {Controller}

    \newpage
    \section {Konzeptionierung einer integrierten Python Anwendung}

    \subsection {Konzepte zur Datenmodellierung}

    \subsection {Konzepte für Controller- und Serviceklassen}

    \subsection {GUI - Konzeptionierung}

    \subsection {Teilautomatisierte Code Dokumentation}

    \newpage
    \section{Analyse zur Fehlerbehandlung}

    \newpage
    \section {Ideensammlung zu kameragestützten Validierungsprozessen in der Lagerverwaltung}

    \subsection {Konzepte}

    \subsection {Abgeleitete Anforderungen an die Kamera}

    \subsection {Kameraauswahl}



\end{document}
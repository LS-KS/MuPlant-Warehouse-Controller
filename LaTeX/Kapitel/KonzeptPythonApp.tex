

\chapter{Konzeptionierung einer integrierten Python Anwendung}\label{PythonApp}

Design Pattern--> Wahl fällt auf MVC-Pattern

\section{PySide6 und QuickQml 2.0}

Laut der Qt Wiki Website \cite{QtWikiHistory} wurde das Qt Framework geboren als ihre Schöpfer Haavard Nord und
Eric Chambe-Eng im Sommer 1990 in Norwegen an einem GUI für eine Ultraschall Datenbank arbeiten.
Die Software sollte damals in C++ implementiert auf Mac, Unix und Windows laufen.
Fünf Jahre später veröffentlichten Sie das erste Qt Framework unter dem Firmennamen Troll Tech.
Seitdem gewann das Framework immer mehr Popularität.
Im Jahr 2006 übernahm Nokia die Firma Trolltech und verkaufte das Qt Project in den Jahren 2011 und 2012 erst teilweise,
dann vollständig an den Digia Konzern.
Seit 2014 ist Qt als Tochterunternehmen des Digia Konzerns unter dem Namen \glqq The Qt Company\grqq ein eigenständiges Unternehmen.

Das Qt Framework ist in C++ implementiert und profitiert dadurch von dem Performance-Vorteil gegenüber anderen
Programmiersprachen.
Die neue Software für die $\mu$Plant soll jedoch in Python implementiert werden.
Für diese Zwecke hat Qt u.A. das Framework PySide6 veröffentlicht, welches einen Wrapper für Python Projekte bietet.

GUI's können in PySide6 im wesentlichen auf zwei Arten erstellt werden.
Eine Möglichkeit ist es, das GUI über sog. Widgets zu erstellen, die direkt im Python Code implementiert werden können.
Die zweite Möglichkeit ist QTQuickQML zu nutzen, bei dem das GUI in einer separaten QML Datei erstellt wird und
mittels einer \verb|QtQMLEngine| Klasse in das Programm eingebunden wird.

Für die Umsetzung eines MVC-Design Patterns empfiehlt sich die Verwendung von QtQuickQML.
Durch die Verwendung des Frameworks wird die konsequente Trennung zwischen Interface und Datenmodell bzw. Services fast
schon erzwungen.
Das behandeln von Events und Kommunikation zwischen dem GUI und dem Backend der Software wird nach dem Signal/Slot Prinzip
gelöst \cite{pysideSignalSlot}.
Dadurch muss man keine zusätzlichen Callbacks oder Lamda-Ausdrücke definieren.
Ein Signal wird an beliebiger Stelle (GUI oder Backend) emittiert und kann an jeder Stelle abgefangen werden.
Eine Funktion die mit der Annotation \glqq Slot(str) \grqq versehen ist, wird als Slot behandelt.
\section{GUI - Konzeptionierung}

\section{Konzepte zur Datenmodellierung}

\section{Konzepte für Controller- und Serviceklassen}

\section{Teilautomatisierte Code Dokumentation}